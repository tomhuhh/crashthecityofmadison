\documentclass[10pt]{article}

\usepackage{tikz}
\usepackage{graphicx}
\usepackage{float}
\usetikzlibrary{mindmap,trees}
\begin{document}


\title{Crash in the City of Madison}
\author{}
\date{}
\maketitle

\section{Demographics}

\large Figure 1 shows the age distribution of drivers involved in crashes in the City of Madison and surrounding suburbs. We can see that the age distribution of drivers involved in crashes aren't very different. As age increases, we see drivers involved in less accidents.

\begin{figure}[H]
\flushleft
\includegraphics[width=150mm]{Age.eps}
\caption{Age Distribution of Drivers in crashes}
\label{1}
\end{figure}

\newpage
\subsection{Is there a correlation between age and drunk driving accidents?}

Figure 2 shows the percentage of drunk drivers in each age group. We can see that 21-25 year-olds have the highest proportion of drunk driving and that crashes are more likely to occur in the suburbs. It is also worth noting that many under-age drivers are also involved in drunk driving, which is a dangerous sign.

\begin{figure}[H]
\flushleft
\includegraphics[width=150mm]{AAge.eps}
\caption{Alcohol-Related Age distribution of Drivers in Crashes}
\label{2}
\end{figure}

\newpage
\section{Gender-based research}

Figure 3 shows the gender composition of drivers involved in crashes in the City of Madison and suburbs. For both regions, the proportion of men is higher than that of women.

\begin{figure}[H]
\flushleft
\includegraphics[width=150mm]{GE.eps}
\caption{Gender Distribution of Drivers in Crashes}
\label{3}
\end{figure}

\newpage
\subsection{Is there a correlation between gender and drunk driving?}

As shown in Figure 4, The percentage of men involved in accidents is above 60\% in both regions. From the results obtained in Figure 3, male drivers accounted for 50 and 60 percent of all crashes in Madison and Suburbs, respectively. It could be inferred that men drunk drive at a higher rate than women.

\begin{figure}[H]
\flushleft
\includegraphics[width=150mm]{Gender.eps}
\caption{Gender Distribution of Drivers in Alcohol-Related Crashes}
\label{4}
\end{figure}

\newpage
\section{Over time}
\subsection{Year}
Figure 5 shows the total number of crashes in the City of Madison v.s. the suburbs from 2000 to 2018. Before 2009, the number of crashes in the Suburbs was always higher than that of the City of Madison. Post 2009, the opposite was the case.

\begin{figure}[H]
\flushleft
\includegraphics[width=150mm]{Yearcrash.eps}
\caption{\# of Crash Distributed by Year}
\label{5}
\end{figure}

\newpage
Figure 6 shows the trends in the percentage of drunk driving in the two regions from 2000 to 2018. We can see that 2010 is a turning point. The ratio of the City of Madison was higher than that of the suburbs before 2010, and the opposite was true after 2010.

\begin{figure}[H]
\flushleft
\includegraphics[width=150mm]{YearcrashA.eps}
\caption{\% Alcohol-Related Crash Distributed by Year}
\label{6}
\end{figure}

\newpage
\subsection{Month}
From Figure 7, we see that there are more crashes occured in cold months.

\begin{figure}[H]
\flushleft
\includegraphics[width=150mm,height=85mm]{Month.eps}
\caption{\# of Crash Distributed by Month}
\label{7}
\end{figure}

\newpage
From Figure 8, we see that Spring and Autumn are the most two dangerous seasons in Surbubs. But for Madison,the drunk driving rate does not fluctuate significantly between seasons.

\begin{figure}[H]
\flushleft
\includegraphics[width=150mm,height=85mm]{MonthA.eps}
\caption{\% Alcohol-Related Crash Distributed by Month}
\label{8}
\end{figure}

\newpage
\subsection{Week}
From Figure 8, we see that as the week goes on, more and more people go out, which correlates to more alcohol-related incidents!
\begin{figure}[H]
\flushleft
\includegraphics[width=150mm,height=85mm]{WeekA.eps}
\caption{\% Alcohol-Related Crash Distributed by Week}
\label{9}
\end{figure}

\newpage
\subsection{Hour}
The peak hours of alcohol-related incidents are from 12 AM till 3 AM. Around 3 AM, we see that there is a bigger percentage of suburbians involved in crashes compared to that of Madison. One theory is that people who live in the suburbs have to drive a longer distance home on the weekends, where buses don't run after 12 AM, which in turn leads to drunk driving.
\begin{figure}[H]
\flushleft
\includegraphics[width=150mm,height=85mm]{HourA.eps}
\caption{\% Alcohol-Related Crash Distributed by Hour}
\label{10}
\end{figure}

\newpage
\subsection{Week and Hour}
From figure 11, we can see that in the City of Madison, the curves on the working day are basically the same, while the curves on the weekend show a completely different pattern. For most of the hours, the number of crashes in working days is greater than the weekend, but in the few hours after the 0:00 AM, the curves of the weekend are higher than that of the working day.

\begin{figure}[H]
\flushleft
\includegraphics[width=150mm,height=85mm]{HourNM.eps}
\caption{\# of Weekly Crash Distributed by Hour in Madison}
\label{11}
\end{figure}

\newpage
For the Surbubs, the number of crashes at 0:00 is unusually high, but the crash distribution at other times are similar to those in the City of Madison. 

\begin{figure}[H]
\flushleft
\includegraphics[width=150mm,height=85mm]{HourNS.eps}
\caption{\# of Crash Distributed by Hour in Surbub}
\label{12}
\end{figure}

\newpage
When we discussed the time distribution of drunk driving, we found that the curve of the working day is similar to the weekend. So you can conclude that the peak hours of drunk driving are between 0:00 and 3:00 AM, regardless of the day of the week.

\begin{figure}[H]
\flushleft
\includegraphics[width=150mm,height=85mm]{HourAM.eps}
\caption{\% Weekly Alcohol-Related Crash Distributed by Hour in Madison}
\label{13}
\end{figure}

\newpage
For the Surbub, the above conclusion is still true.

\begin{figure}[H]
\flushleft
\includegraphics[width=150mm,height=85mm]{HourAS.eps}
\caption{\% Weekly Alcohol-Related Crash Distributed by Hour in Surbub}
\label{14}
\end{figure}

\newpage
\section{About the types of crash}

This pie chart shows the incidence of all accident types in Dane County.

\begin{figure}[H]
\flushleft
\includegraphics[width=150mm]{ACCDTYPE.eps}
\caption{Types of Crash}
\label{15}
\end{figure}

\newpage
Because almost every crash has a personal injury, we next analyze the fatal accidents in the City of Madison and the suburbs.
Figure 8 shows the incidence of fatal events in different crash types (ten types with the highest lethality) in the City of Madison.

\begin{figure}[H]
\flushleft
\includegraphics[width=130mm]{TYPEF.eps}
\caption{\% Fatality Distributed by Type in Madison}
\label{16}
\end{figure}

\newpage
Figure 9 shows the incidence of fatal events in different crash types (ten types with the highest lethality) in the Suburbs. We can see the difference between the two regions, which shows that different approaches must be adopted for fatal accidents in these two regions.

\begin{figure}[H]
\flushleft
\includegraphics[width=130mm]{TYPEFS.eps}
\caption{\% Fatality Distributed by Type in Suburbs}
\label{17}
\end{figure}

\newpage
\section{Severity}
\subsection{Speed}
Table 1 shows the probability of crashes of different severity at different speed limits.

% Table generated by Excel2LaTeX from sheet 'Sheet1'
\begin{table}[H]
  \centering
  \caption{The Relationship between Speed Limit and Severity}
    \begin{tabular}{rrrrrrrrr}
    \multicolumn{2}{c}{Fatal} & \multicolumn{2}{c}{Incapacitating} & \multicolumn{2}{c}{Non-incapacitating} & \multicolumn{2}{c}{Unknown} & \multicolumn{1}{l}{Total} \\
    \hline
    \multicolumn{1}{c}{\#} & \multicolumn{1}{c}{\%} & \multicolumn{1}{c}{\#} & \multicolumn{1}{c}{\%} & \multicolumn{1}{c}{\#} & \multicolumn{1}{c}{\%} & \multicolumn{1}{c}{\#} & \multicolumn{1}{c}{\%} & \multicolumn{1}{c}{\#} \\
    \hline
    37    & 0.12\% & 557   & 1.81\% & 2991  & 9.73\% & 27140 & 88.33\% & 30725 \\
    59    & 0.16\% & 588   & 1.64\% & 3422  & 9.55\% & 31753 & 88.64\% & 35822 \\
    34    & 0.34\% & 196   & 1.94\% & 960   & 9.50\% & 8912  & 88.22\% & 10102 \\
    13    & 0.35\% & 69    & 1.87\% & 336   & 9.11\% & 3271  & 88.67\% & 3689 \\
    \hline
    \end{tabular}%
  \label{tab:addlabel}%
\end{table}%

Figure 10 shows the number of crashes in different speed limits at three different severity levels. We can see that for fatal accidents, the higher the speed limit, the greater the number. For minor injuries that are non-incapacitating, the lower the speed limit, the greater the number. For incapacitating accidents, there is no significant difference in the amount of occurrence between different speed limits.

\begin{figure}[H]
\flushleft
\includegraphics[width=150mm,height=80mm]{SPEED.eps}
\caption{The Correlation between Speed Limit and Severity}
\label{18}
\end{figure}

\newpage
\subsection{Age}

We can see that the percentage of minor injuries and disability between different age groups fluctuates little.

\begin{figure}[H]
\flushleft
\includegraphics[width=150mm,height=100mm]{AgeS.eps}
\caption{\% Severity Distributed by Age}
\label{19}
\end{figure}

\newpage

From Figure 20, we can see that the percentage of the disability accident in the drunk driving is higher than the above picture (all crahes) and the percentage of the non-disability accident is lower than the above picture (all crahes), so the drunk driving accident will bring more serious damage to the driver.

\begin{figure}[H]
\flushleft
\includegraphics[width=150mm,height=100mm]{AgeSA.eps}
\caption{\% Alcohol-Related Severity Distributed by Age}
\label{20}
\end{figure}

\newpage
From figure 21, we can see that among all accidents,crashes involving drivers aged 45-55 have the lowest fatality rates.

\begin{figure}[H]
\flushleft
\includegraphics[width=150mm,height=100mm]{AgeSF.eps}
\caption{\% Fatality Distributed by Age}
\label{21}
\end{figure}

\newpage
But in the case of drunk driving, crashes involving drivers aged 45-55 have the highest fatality rates.

\begin{figure}[H]
\flushleft
\includegraphics[width=150mm,height=100mm]{AgeSFA.eps}
\caption{\% Alcohol-Related Fatality Distributed by Age}
\label{22}
\end{figure}

\end{document}